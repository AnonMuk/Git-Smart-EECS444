\documentclass[]{acmart}
\usepackage{booktabs}
\usepackage{listings}
\usepackage{color}
\usepackage{graphicx}
\usepackage[utf8]{inputenc}

\title{Analysis of GitHub and Stack Overflow Code for Malicious Content}
\acmDOI {}
\acmMonth{12}
\acmSubmissionID{}
\acmYear{2019}
\author{Steven Nyeo}
\affiliation{%
\institution{Case Western Reserve University}
\department {Department of Electrical, Computer, and Systems Engineering}
\city{Cleveland}
\state{Ohio}
\postcode{44106}
\country{USA}}
\email{cxn152@case.edu}
\author {Patrick Hogrell}
\affiliation{%
\institution{Case Western Reserve University}
\department {Department of Computer and Data Sciences}
\city{Cleveland}
\state{Ohio}
\postcode{44106}
\country{USA}}
\email{pjh96@case.edu}
\author{Zubair Mukhi}
\affiliation{%
\institution{Case Western Reserve University}
\department {Department of Computer and Data Sciences}
\city{Cleveland}
\state{Ohio}
\postcode{44106}
\country{USA}}
\email{zxm132@case.edu}
\author{Chris Shorter}
\affiliation{%
\institution{Case Western Reserve University}
\department {College of Arts and Sciences}
\city{Cleveland}
\state{Ohio}
\postcode{44106}
\country{USA}}
\email{cws68@case.edu}
\date{December 3, 2019}

\begin{document}

\maketitle
\tableofcontents
\section*{Acknowledgements}
We would like to acknowledge Professor Yanfang (Fanny) Ye and VirusTotal for their assistance in this project. 
\section*{Abstract}
The Internet and specifically code repositories serve as a space for the spread of solutions to complex software problems, computing research, and software libraries and executables that decrease reproduction of labor. One of the most popular repository sites is \href{https://github.com}{GitHub}. GitHub, however, has limited protection for end-users, as GitHub only recently started adding code screening this year\footnote{See Literature Review, The Supreme Backdoor Factory}. For example, a malicious GitHub repository posing as a benign solution to a common but time-intensive coding problem could easily disseminate malicious code or poor programming practices that allow for bad actors to compromise otherwise safe systems. Additionally, malware and exploits are occasionally hosted on GitHub\footnote{see \url{https://github.com/frohoff/ysoserial} as an example of a proof-of-concept exploit hosted on GitHub}. While these files are often made public for the purpose of full disclosure, there is still a potential that a popular fork of the project can contain malware. To resolve this, we built an automated webscraper and an analysis tool to locate and analyze GitHub projects and ensure their security.
\section*{Motivation and Background}
\section*{Literature Survey}
\section*{Methodology}
\subsection*{Repository Scraping}
\subsection*{Repository Scanning}
\section*{Results}
\subsection*{Overall Findings}
\subsection*{Key Players}
\subsubsection*{Player 1}
\section*{Countermeasures}
\section*{Conclusions}
\end{document}
