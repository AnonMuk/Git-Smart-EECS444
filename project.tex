\documentclass[]{acmart}
\usepackage{booktabs}
\usepackage{listings}
\usepackage{color}
\usepackage{graphicx}
\usepackage[utf8]{inputenc}

\title{Analysis of GitHub and Stack Overflow Code for Malicious Content}
\acmDOI {}
\acmMonth{12}
\acmSubmissionID{}
\acmYear{2019}
\author{Steven Nyeo}
\affiliation{%
\institution{Case Western Reserve University}
\department {Department of Computer and Data Sciences}
\city{Cleveland}
\state{Ohio}
\postcode{44106}
\country{USA}}
\email{cxn152@case.edu}
\author {Patrick Hogrell}
\affiliation{%
\institution{Case Western Reserve University}
\department {Department of Computer and Data Sciences}
\city{Cleveland}
\state{Ohio}
\postcode{44106}
\country{USA}}
\email{pjh96@case.edu}
\author{Zubair Mukhi}
\affiliation{%
\institution{Case Western Reserve University}
\department {Department of Computer and Data Sciences}
\city{Cleveland}
\state{Ohio}
\postcode{44106}
\country{USA}}
\email{zxm132@case.edu}
\author{Chris Shorter}
\affiliation{%
\institution{Case Western Reserve University}
\department {College of Arts and Sciences}
\city{Cleveland}
\state{Ohio}
\postcode{44106}
\country{USA}}
\email{cws68@case.edu}
\date{December 3, 2019}

\begin{document}

\maketitle
\tableofcontents
\section*{Acknowledgements}
We would like to acknowledge Professor Yanfang (Fanny) Ye and VirusTotal for their assistance in this project. 
\section*{Abstract}
Cyberspace serves as a space for the spread of research, information, and solutions to problems that individuals are unable to solve for whatever reason. Some spaces to provide these solutions are StackOverflow and GitHub. These sources, however, have no checks against malicious solutions. For example, a malicious GitHub repository posing as a benign solution to a common but time-intensive problem could easily disseminate malicious code or poor programming practices that allow for bad actors to compromise otherwise safe systems. Additionally, malware and exploits are occasionally hosted on GitHub (see, for example, \url{https://github.com/frohoff/ysoserial} as an example of potential exploits hosted on GitHub). While these are often made public for the purpose of full disclosure, there is still a potential that a popular fork of the project can contain malware. To resolve this, we built an automated webscraper and an analysis tool to locate and analyze popular GitHub projects and ensure their security.
\section*{Methodology}
\end{document}
